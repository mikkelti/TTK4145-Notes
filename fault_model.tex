\part{Fault model \& software fault masking}
Keywords from the part on basic fault tolerance are:
\begin{itemize}
    \item Acceptance tests
    \item Merging failure modes
    \item Redundancy
\end{itemize}
For a fault tolerant system we want the following (progressively better):
\begin{enumerate}
    \item Failfast (errors are detected immediately)
    \item Reliable (failfast + the system is repaired)
    \item Available (continuous operation)
\end{enumerate}
An overview of the process is:
\begin{enumerate}
    \item Find failure modes.
    \item Detect errors / simplify error model, inject errors for testing
    \item Error handling by redundancy $\rightarrow$ reliable and available module.
\end{enumerate}
On error injection:
\begin{enumerate}
    \item Simplify by merging failure modes.
    \item Inject failed acceptance tests.
\end{enumerate}
Examples for fault tolerant modules, three basic modules: storage, communication and processing.
\section{Storage}
Imagine an array of data. Assume unreliable read and write functions.
\subsection{Failure modes}
For writing we have:
\begin{itemize}
    \item Writes the wrong data
    \item Writes to the wrong place
    \item Does not write
    \item Fails
\end{itemize}
Likewise, for reading:
\begin{itemize}
    \item Give wrong data
    \item Give old data
    \item Give data from wrong place
    \item Fails
\end{itemize}

\subsection{Detection, merging of error modes and error injecction}
\begin{itemize}
    \item Detect by also writing address, checksum, versionID and statusbit to the buffer.
    \item Merging: All errors $\rightarrow$ Fail
    \item For error injection, spawn a thread that runs in parallel and flips status bits. This seems really smart!
\end{itemize}
\subsection{Handling with redundancy}
\begin{itemize}
    \item Keep several copies of the buffer, the one with the newest versionID is used (returned).
    \item Always write back when a reading error occurs (write a 'safe state'?).
    \item 
\end{itemize}
\section{Messages}
\subsection{Failure modes}
\begin{itemize}
    \item Lost
    \item Delayed
    \item Corrupted
    \item Duplicated
    \item Wrong recipient
\end{itemize}

\subsection{Detection, merging of error modes}
\begin{itemize}
    \item Session ID
    \item Checksum
    \item Ack (acknowledgement)
    \item Sequence numbers
    \item All errors $\rightarrow$ Lost message
\end{itemize}

\subsection{Handling with redundancy}
\begin{itemize}
    \item Timeout and retransmission
\end{itemize}

\section{Processes/calculations}
\textbf{Error mode:} Does not yield the next correct 'side effect'.\\
\textbf{Detect and merge:} All failed acceptance tests $\rightarrow$ STOP (failfast).\\
Three ways to handle with redundancy, described in the following subsections.

\subsection{Checkpoint-restart}
\begin{itemize}
    \item Write state to storage after each (successful) acceptance test, before each side effect.
    \item This yields error containment, but requires good acceptance tests.
\end{itemize}
\subsection{Process pairs}
\begin{itemize}
    \item Two processes; primary and backup (primary does the work).
    \item Backup takes over if (when) primary fails, and becomes new primary.
    \item Primary sends IAmAlive-messages and checkpoints to backup.
    \item NB! Because of resending this does not rely on safe communication. I.e. we get redundancy by resending, communication errors are masked out.
\end{itemize}

\subsection{Persistent processes}
\begin{itemize}
    \item Transactional infrastructure
    \item All calculations are transactions, i.e. atomic transformations from one consistent state to another.
    \item The processes are then 'stateless', all states are stored in a database.
    \item For such simple processes OS can take care of restart.
\end{itemize}
Note that reliable and available storage, communication and calculations are necessary to make this transactional infrastructure. Hence one often falls back to one of the first two options.
