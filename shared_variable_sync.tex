\part{Shared variable synchronization}
\section{Semaphores}
A \textbf{semaphore} is like an integer, with three key differences:
\begin{enumerate}
    \item It can be initialized to any value, but after that only incremented or decremented (its value cannot be read).
    \item When the semaphore is decremented by a thread, and the result is negative, the thread blocks.
    \item When the semaphore is incremented by a thread, one waiting thread gets unblocked (if any).
\end{enumerate} 
Things to note:
\begin{itemize}
    \item After one thread increments the semaphore, and another is woken, they run concurrently.
    \item A positive value represents the number of threads that can decrement without blocking.
    \item A negative number represents the number threads that have blocked and are waiting.
\end{itemize}
The basic syntax used is:
\begin{itemize}
    \item \texttt{sem = Semaphore(1)} to create a new semaphore with the given initial value.
    \item \texttt{sem.signal()} to increment the semaphore (and wake a waiting thread).
    \item \texttt{sem.wait()} to decrement the semaphore (and block if the result is negative).
\end{itemize}
In the classical example of incrementing/decrementing \texttt{int i}, \texttt{i}'s value must be set pretty high to see synchronization errors. The reason for this is that context switching (switching between threads and saving states) does not happen until a certain time passes. For small \texttt{i}, one thread will typically finish before the second gets a chance to start.

\section{Standard problems}
A (data) \textbf{race condition} is a fault in the design of the interaction between two threads, which leads to the result being highly dependent on the sequence and timing of access to shared variables.

A \textbf{livelock} occurs when a thread gets stuck in e.g. a busy-waiting loop (more generally; a subset of the possible states), and is unable to proceed.

A \textbf{deadlock} occurs when the system is stuck in a circular wait, with no threads being able to proceed. 

\textbf{Starvation} happens when a thread is unable to gain access to a resource it requires, because other threads keep getting it first. Can be caused by an 'unfair' scheduler.

\section{Monitors}
What is bad with semaphores?
\begin{itemize}
    \item Forgetting a wait can lead to multplie threads running concurrently in a \textbf{critical region} (a section of code that should always be executed under mutual exclusion).
    \item Forgetting a signal can lead to a deadlock.
    \item The code is distributed all over the system, making maintenance hard.
\end{itemize}

\subsection{Critical regions}
\begin{itemize}
    \item Critical regions but the responsibility for mutual exclusion on the compiler, instead of the programmer. Critical regions of the same name mutually exclude.
    \item In \textbf{conditional critical regions}, the thread must wait for the mutex lock \emph{and} check a boolean condition before it can proceed. A \textbf{guard} is a more sophisticated form of this boolean condition, which instead runs a test (procedure) that opens or closes for execution.
\end{itemize}
Limitations
\begin{itemize}
    \item (Conditional) critical regions are still distributed throughout the program.
\end{itemize}

\subsection{Monitors}
\begin{itemize}
    \item A collection of local variables and procedures (basically a module), with a mutex that only allows one thread at a time to access its methods and variables.
    \item Pro: All code that accesses the shared data is localized.
    \item Has \textbf{condition variables} (queues of threads waiting for some condition to be true) with operations \texttt{suspend} and \texttt{resume} to block and unblock threads.
    \item \texttt{suspend} releases the monitor lock.
    \item The operations are safe, because they can only be accessed/called from inside the monitor.
\end{itemize}
Cons
\begin{itemize}
    \item Suspending in a nested call does not release the outer lock. Ex. a procedure in monitor A calls a procedure in monitor B, which contains a \texttt{suspend} statement. This makes it infeasible to build supermodules from submodules (that are monitors). 
    \item The possibility for deadlocks and data races still exists.
    \item Does not provide other synchronization than condition variables.
\end{itemize}

\section{Synchronization mechanisms in POSIX, Java and Ada}
\subsection{Java}
The \texttt{synchronized} keyword is used to make methods thread-safe. Consider the following class:
\begin{minted}{java}
    public class MyClass {
        private int i;

        public MyClass(int initValue) {
            i = initValue;
        }

        public synchronized void increment() {
            i++;
        }
    }
\end{minted}
When thread A is executing \texttt{increment()}, all other threads that (wish to) invoke synchronized methods for the same \texttt{MyClass}-object block (suspend) until thread A releases the monitor lock.

\texttt{wait()} is used to suspend the current thread, like this:
\begin{minted}{java}
    public synchronized void conditionalIncrement() {
        while(i < 3) wait();
        i++;
    }
\end{minted}
Note that since the method is \texttt{synchronized}, thread A must hold the monitor lock before it can invoke it. When it calls \texttt{wait()}, it releases the lock and suspends execution. It will be woken up at some time after the following procedure has run:
\begin{minted}{java}
    public synchronized void importantChange() {
        i = 3;
        notifyAll();
    }
\end{minted} 
All threads waiting on the lock are notified when \texttt{notifyAll()} is called.

\subsection{Ada}
Ada offers monitors as \emph{protected modules}. A protected module can be either a single protected object or a protected type, which can contain many objects. 
